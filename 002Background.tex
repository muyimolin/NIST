
%-------------------------------------------------------------------------
\section{Background}\label{sec:back}
%-------------------------------------------------------------------------

%-------------------------------------------------------------------------
\subsection{Performance measurement in robot teleoperation}\label{sec:back-performance}
%-------------------------------------------------------------------------
In teleoperation framework, human control a robot through a teleoperation interface to interact with the world. 
Measurement metrics have been developed to evaluate the performance of the four components: human, interface, robot and task, yet these performance metrics haven’t been united in a framework that reflects the adaption level  of human motor control to the physical capability of the robotic system being teleoperated. 
As a result, the performance metrics for human-robot teleoperation system have been developed for each individual component, without little consideration of how they influence each other in a system. For instance, the performance of interface and robot are mostly measured by the motion and perception capability of hardware. Task performance are generally measured completion time and error rate, with domain-specific details. Human performance measurement focuses on the fatigue, stress, attention, and engagement during the tasks. Subjective measurement generally follows the guideline of NASA TXL, while objective performance can measured by or inferred from physiological parameters, kinematic and dynamic motion variables, gaze, muscle and brain activities. 
However, none of these metrics considers human-robot teleoperation as the skill of human using a robotic tool. From this perspective, human performance should be evaluated by the capability of discovering, learning, and refining the motion primitives can be performed by the teleoperated robot, and the capability of task planning and re-planning using the acquired motion primitives. On the other hand, the usability of teleoperation interface should be evaluated by how well it can convey and assist intended human motion and reduce the complexity of controlling motion coordination. 

%-------------------------------------------------------------------------
\subsection{From Teleoperation to Intelligent Assistance}\label{sec:back-intelligent}
%-------------------------------------------------------------------------

Similar to the industrial revolution, task automation has been transforming the interaction and collaboration of human workers and AI agents in healthcare, industry, social service and military tasks. This ``white-collar revolution`` has significantly changed the paradigm of human-robot teleoperation. On one hand, intelligent assistive robots augment the physical and cognitive capabilities of human workers, relieve them from repetitive, tedious, effort-demanding, and dangerous tasks. On the other hand, the-state-of-the-art intelligent robot cannot perform complex and risk-sensitive task without human control. As the tasks for robots increasing with the robot capabilities, teleoperation, with more or less task automation, will remain to be the most reliable way to perform the most difficult tasks. It is also the most efficient way to synergize human-robot performance, and explore and utilize robot physical capability. 
In this context, performance metrics for teleoperation system have been developed to inform interface design and worker skill acquisition process. Robot control interfaces need to provide intelligent assistance to the tasks hard or tedious for direct human control, yet easy, efficient and safe for robot to automate. Study on the robot motion coordination, teleoperated via various teleoperation interface by novice and expert teleoperators, can help to identify such tasks. For routine tasks, motion coordination can be learned and automated through the combination of imitation and reinforcement learning. For free-style tasks, robot teleoperators have been assisted with motion smoothing and scaling to improve their precision control, and with power augmentation in labor-demanding work. Little research has considered reducing the control effort by automating secondary tasks and the control of the redundant degrees of freedom in motion coordination. 

%-------------------------------------------------------------------------
\subsection{Adaption of Human Motor Control to Robotic Physical Embodiment}\label{sec:back-intelligent}
%-------------------------------------------------------------------------
The control effort-sharing between robot automation and direct human control can refer to how human motor system regulate the controlled and uncontrolled manifolds. Take reaching-to-grasp motion for instance, as human focusing on control hand position and orientation, the arm posture (i.e., how to pose the elbow) is automatically handled. In bimanual coordination, human focuses more on the control of dominant hand, while dominant hand follows naturally and accordingly. Intelligent robot control interface needs to automate these naturally uncontrolled and secondary tasks, particularly for complex robotic systems. Providing such intelligent assistance also reduce the learning efforts in teleoperation skill acquisition. Learning to control a robot through teleoperation interface is essentially a process of adapting human motor control to a robot physical embodiment. Ideally, a teleoperation interface should minimize the adaption efforts. One way to achieve that is to ensure the motion coordination rendered under shared autonomous control comply with the high-level motion control strategy of human motion. These high-level motion control strategies are abstract from human/robot physical embodiment (unlike the low-level coordination such as joint coupling), and reflect the rationality of how physical structure, perception and motion control synergize with each other. The high-level coordination strategies that involve both robot motion and perception haven’t well-studied. Related work on human motion coordination (e.g., arm redundancy, arm-hand coordination, synergy in hand motion, gait cycle and whole-body coordination in locomotion, etc), robot motion coordination (e.g., coordination of macro-micro manipulators, mobile manipulator robots, humanoid robot, etc), and interactive-perception can be united to enhance the understanding. 

% %-------------------------------------------------------------------------
% \subsection{PI's related work}\label{sec:back-PI}
% %-------------------------------------------------------------------------


