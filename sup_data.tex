\pagebreak

\begin{center}
	{\Large \bf Supplementary Documents --- Data Management}\\
%     \vspace{4pt}
% 	\renewcommand{\baselinestretch}{1}
%    	{\large PI: Zhi Li (Worcester Polytechnic Institute)\\
%     Co-PI: Brian Ziebart (University of Illinois at Chicago), Jie Fu (Worcester Polytechnic Institute)}
   	% \vspace{4pt}
    % {\large Worcester Polytechnic Institute}
\end{center}

\vspace{0.5em}
%-------------------------------------------------------------------------
\section{Roles and responsibilities}\label{sec:resp}
%-------------------------------------------------------------------------

PI Li, Fu and Ziebart (referred as “the PI” in this document) will be jointly responsible for data management and for monitoring the data management plan. They and their PhD students working on the proposed project under their supervision will adherence to this data management plan be checked or demonstrated. They have responsibility over time for decisions about the data once the PhD students are no longer available.

%-------------------------------------------------------------------------
\section{Types of Data and Policies for Storage}\label{sec:type}
%-------------------------------------------------------------------------
The data and materials expected to be produced will include computational codes, software, raw data files from experiments, data analysis files, simulation data, video recordings of experiments, etc. Each class of data is described below:
\begin{itemize}	

\item Experimental data: The data collected in the experimental user study include: 
\begin{itemize}
\item Kinematic motion data, i.e., the 3-dimensional position of the passive reflective markers attached to the human subjects recorded using a Vicon motion capture system and saved in .csv and .c3d formats.
\item Digital video and audio data that record the experiment trials.
\item Subjects’ biometric information for data analysis, such as age, gender, height, and weight.
\item Subjects’ personal information for contact, including name, phone number, and email.
\end{itemize}

Data from each subject will be stored in password-protected computers under the subject ID automatically generated by the motion capture system. Access to the data is limited to faculty and students working on the proposed project and requires the PI’s permission. The retained copy of the videotape of subjects’ motion will be processed so that the faces of the subjects are blurred to be unrecognizable, and will only be disseminated with the consent of the subject in the video. Hard copies of the questionnaires and consent forms will be shredded and discarded. The motion data, video, audio, and subject’s biometric information will be used for research purposes and retained indefinitely. Personal information will be kept confidential and will be discarded as soon as the experiments are completed. No identifiable information will be recorded or used for the purpose of study.

\item Computational codes and software development data

This data includes Matlab, OpenSim, Robot Operating System (ROS), and customized robot control software codes (in C/C++/Python languages) for simulations and experiments. During the development, the data will be kept in a private repository managed by IT Service of Worcester Polytechnic Institute. The developed software will be maintained and accessible to the public for non-profit research after post-processing and verification of its correctness. The data will be in standard, non-proprietary file formats to facilitate both data sharing and long-term preservation.

\item Documentation for research and education activities

The research results of the proposed project will be disseminated via academic reports, software documentation and instruction manuals, lecture notes and slides, presentations, and publications. The data are saved in \LaTeX files, image files, Word/Excel/Power Point documents, and PDF files. 
\end{itemize}

\noindent
Data and metadata standards will use common ROS encodings for datasets. If any non-standard formats are used, documentation on how to use and access the data will be provided.
% PM: Changed this a bit, double-check?

%-------------------------------------------------------------------------
\section{Methods and policies for providing access and enabling sharing}\label{sec:access}
%-------------------------------------------------------------------------

The data produced in this project will not be confidential but will require the PI’s permission to access. Experimental data will be accessible only to the PI’s lab members and collaborators, while access to computational codes and software development data can be granted for non-profit academic research. Documentation for research and education activities will be available through both authors’ websites and the digital archives of the corresponding publishers. No delays in publication of the research outcomes will occur. 

%-------------------------------------------------------------------------
\section{Policies and provisions for re-use and re-distribution}\label{sec:reuse}
%-------------------------------------------------------------------------
Researchers who request the data can reuse it within the research community. The data cannot be used for commercial applications or purposes and may not be changed/resubmitted without the PI’s permission. The data is subject to WPI’s intellectual property policies.

%-------------------------------------------------------------------------
\section{Plans for archiving and preservation of access}\label{sec:axiv}
%-------------------------------------------------------------------------
The project website along with all the related material and data components will be stored on WPI servers. WPI is committed to its continuous maintenance, availability, and backups. The servers are equipped with primary and backup power, environmental, and communication systems. Physical access to this room is strictly limited to current WPI systems administrators by means of electronic locks. An alarm system that is constantly monitored by campus police has been placed to control access to the room and eliminate theft.