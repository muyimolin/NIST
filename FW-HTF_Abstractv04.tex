\pagebreak

\begin{center}
	{\Large \bf FW-HTF Theme 2: \textbf{Toward the Next Generation of Nursing Technologies}}\\
    \vspace{4pt}
% 	\renewcommand{\baselinestretch}{1}
   	{\large PI: Zhi Li, Co-PI: Jeanine Skorinko (Worcester Polytechnic Institute)\\
    Co-PI: Jacob Rosen (University of California, Los Angles)}
   	% \vspace{4pt}
    % {\large Worcester Polytechnic Institute}
\end{center}

% \vspace{1 em}

\paragraph*{\Large Project Summary} 
% Tele-operated robotic systems extend a human worker’s physical capabilities to perform manufacturing and maintenance tasks in remote, inaccessible, dangerous and/or hazardous environments. 
This project aims to (1) develop novel teleoperation interface and training methodology that facilitate healthcare workers to acquire the motor skills of controlling complex motion coordinations frequently performed in nursing and assisting tasks; and (2) investigate the impacts of the proposed technology and training paradigm on healthcare jobs given the socio-cultural norms and biases. The state-of-the-art tele-nursing robots have been endowed with the physical structures for performing arm-hand coordination, bimanual coordination and loco-manipulation while perceiving environment through multi-model sensors from various perspectives. However, these capabilities haven't been exploited due the difficulties of learning the motion and perception mapping imposed by teleoperation interface. Measurement metrics for human and robot performances are not sufficient to quantify the characteristics of teleoperation interfaces, and compare the synergistic human-robot performance through the interfaces across tasks and along the user's motor skills progression. Novel training paradigm to adapt to the technology advancement hasn't been investigated in nursing education. On the social-economical side, it is unclear if the novel tele-nursing technology will significantly shift the job market due to socio-cultural biases (e.g., gender, age, etc) and the correlated education barriers. The socio-cultural biases may interfere with the design of the device and also the desire and ability to learn and use (tele)operation interfaces. For instance, it is possible that during the design process differences between genders are not examined. It is also possible that women are less interested (or even less able) to use (tele)operation interfaces due to gender bias or stereotype threat. Likewise, patients may be less likely to trust a tele-operated robotic system if it controlled by a female compared to a male.  

To address the above technological, educational and social-economical issues, we propose to synergize research efforts from robotics, nursing education and Social psychology to achieve the following research objectives. Our Science and technology objectives aim to (1) Develop novel metrics to characterize the teleoperation interface (the operation complexity, motion mapping intuitivity, efficiency and predictability, perception transparency and balance, etc., and compare various teleoperation interfaces developed for a mobile humanoid nursing robots in motion coordination tasks. Based on the understanding of human-robot adaption through the teleoperation interfaces, we will further develop technologies that can (2) Shift the boundary between direct teleoperation and autonomous control based on the physical and mental status of the operator, (3) synthesize and convert sensory information to improve cognitive situation-awareness, and improve cognitive and physical skills of the operator, (4) Infer human teleoperator's contextual intent based on the knowledge of manipulation tasks and human motions, in order to automate appropriate low-level robot actions.  Our education objective aims to (1) Integrate ``cloud wisdom'' of multiple intelligence agents (human experts and AIs) into the skill evaluation and collaborative decision-making in terms cognitive augmentation, (2) Utilize interactive perception to engage the learning user to actively explore the robot's motion and perception capabilities. Our social-economy objectives aim to (1) determine nursing tasks where a tele-operated robotic system will be of assistance and design robotic system, (2) investigate perceptions of tele-operated robotic systems in relation to socio-cultural norms and biases to assist in the design and implementation. 

% Complex and risk-sensitive tasks that require human-level manipulation dexterity and decision-making intelligence are infeasible through autonomous control, yet can be accomplished under direct (tele)operation.

\vspace{0.5 em}

\paragraph*{\Large Intellectual Merit}
This project addresses how multi-modality teleoperation interface affects the adaption of human motor behavior adapt to the motion and perception capabilities of a robotic physical embodiment. It develops novel methodologies to evaluate the adaption level by the quality of low-level motor skills and high-level task plan. It uniquely integrates collaborative skill evaluation method to teleoperation skill acquisition process, and guides the learning process by maximizing the expected perception and motion information gain.  Moreover, this will be some of the first work to explicitly examine the effects that gender has on: a) perceptions of tele-operated robotic systems, individuals desire and ability to use tele-operated systems, and b) perceptions of the competence of the user of the tele-robotic system. It develops a unified framework that takes into consideration gender bias, stereotype threat, and other socio-cultural norms that may be at play, which will further influence the technology development and worker education. 

% to robot learning that acquire motion and task knowledge through  (e.g., reinforcement learning with explicit reward function v.s. learning through convolutional neural network). Inspired by how human can develop situational awareness and motor skills through intuitive and abstract cognitive feedback and augmentation, we will further develop novel robot teaching methodologies for that leverage human-guided robot interactions with environments and demonstrations/critiques from human teachers. 


% By investigating the underlying principles of the cognitive and physical interaction between the human worker and teleoperation interfaces, we will advance the technologies for user-adaptive cognitive augmentation to facilitate novice workers to build their situational awareness and master the motion mapping between the input interfaces and their physical embodiments in the tasks. Towards the seamless integration of human and (tele-)operated robotic systems, our research will further investigate shared-autonomous control methods for adjusting the human-robot control efforts based on human worker's skill level and physical/mental states. 

% We will also investigate how to utilize the teleoperation interface as an efficient robot teaching interface, to relieve workers from repetitive low-level teleoperation. 


% Towards the seamless integration of human and (tele-)operated robotic systems, we will further investigate shared-autonomous control methods for adjusting the human-robot control efforts based on human worker's skill level and physical/mental states. We will also investigate how to utilize the teleoperation interface as an efficient robot teaching interface, to relieve workers from repetitive low-level teleoperation. 

% This project aims to discover the new knowledge of how multi-modality cognitive feedback affects explicit and implicit learning processes of coordinated manipulation motor skills, and develop novel methodologies for 

% how this knowledge can be used to (1) 

% develop robot control interface that integrates intuitive and abstract multi-modality  to provide user-adaptive   

% Specifically, we will experimentally study in teleoperated manipulation tasks, how human decision-making and task operation can be affected by single- and multi-modality cognitive feedback, and investigate methods for intuitive and integrated representation of task information and cognitive feedback, to reduce an expert worker's the physical and mental efforts. For novice workers, we will focus on how multi-modality cognitive feedback affects the explicit and implicit learning of dexterous and coordinated manipulation motor skills, as well as when and how to provide high-level/abstract cognitive feedback (e.g., verbal/text instructions, numbers, etc.) and low-level intuitive cognitive feedback (e.g., colors, shapes, sounds, tactile and forces, etc.) to facilitate the interactive and associated explicit and implicit motro learning. By investigating the underlying principles of the interaction between the human worker and teleoperation interfaces, we will develop user-adaptive cognitive augmentation to facilitate novice workers to build their situational awareness and master the motion mapping between the input interfaces and their physical embodiments in the tasks. Towards the seamless integration of human and (tele-)operated robotic systems, we will further investigate shared-autonomous control methods for adjusting the human-robot control efforts based on human worker's skill level and physical/mental states. We will also investigate how to utilize the teleoperation interface as an efficient robot teaching interface, to relieve workers from repetitive low-level teleoperation. 



\vspace{0.5 em}

\paragraph*{\Large Broader Impacts}
Our research will directly and primarily benefit healthcare workers that operate tele-nursing robotic system, and will have boarder impacts on a wide range of workers that need to master the skill of operating complex human-machine interfaces for direct control and robot teaching, such as tele-surgery and tele-manufacturing. In addition, this work will explicitly examine the effects that gender of the perceiver and user have on perceptions of tele-operated robotic systems and ability to use the system. By improving the usability of teleoperation interface with gender as consideration, we aim to improve the availability of healthcare, industrial, and social service labor, and provide capable surrogates for military and medical personnel in tedious, repetitive, and dangerous tasks. It will also lead towards affordable robotic solutions can provide assistance and job opportunities to wider populations. Our research will also synergize with graduate and undergraduate education for students from engineering, medical, and nursing schools, and will actively engage K-12 students and the general public. 

% Through robot-mediated interactions, develops robot motion intelligence in a multi-agent, highly-interactive context. 



% to navigate in cluttered human environments and perform a wide variety of dexterous manipulation tasks with minimal human control. Our key idea is to develop a unified framework for lifelong learning and fast, context-based intent and in simultaneous multi-lateral physical human-robot interactions. In such scenarios, the nursing robot participates in a patient-caring task, while learning when and how to intervene in the robot-mediated collaboration between its remote teleoperator and on-site human partners. By observing human experts, the nursing robot will also establish hierarchical knowledge of natural coordinated human motions and human-human interactions, and metrics for evaluating task performance and motion capabilities. Such motion knowledge and metrics will be used to evaluate the level of skills of novice teleoperators, patients and partner nurses, and adjust the level of assistance provided to maintain nursing task performance and fluency of human-robot collaboration. 

  
% \vspace{0.5 em}

% \paragraph*{\Large Intellectual Merit}
% Our proposal aims to address the need for \textit{\textbf{customizable robot motion intelligence}}, and to \textit{\textbf{lower the barriers}} for medical personnel to synergize with tele-robotic technologies. Our proposed framework enables a tele-robotic system to develop and evolve its contextual motion intelligence through lifelong interactions with human experts, and apply its motion intelligence to provide user-adaptive assistance to reduce learning and operation effort for novice teleoperators.   



% \vspace{0.5 em}

% \paragraph*{\Large Broader Impacts}
% This project envisions broader impacts on a wide range of mobile humanoid robots for medical, industrial, and social service tasks. Our research efforts will enable these robots to interact simultaneously and physically with both the teleoperator and end users, and reconcile their intents to achieve natural, fluent, and intimate collaboration.
% % PM: I changed this to singular "barrier" to make the claim a bit less grand? Feel free to change back if desired.
% We aim to remove a major barrier that prevents robots from integrating into human society as capable and socially acceptable peers. By improving the usability of dexterous robotic manipulators under direct teleoperation and shared-autonomous control, this project may also lead to improved availability of healthcare, industrial, and social service labor, and provide surrogates for military and medical personnel for tedious, repetitive, and dangerous tasks. It will lead towards affordable robotic solutions for hospital and home care that can provide long-term assistance to aging and disabled populations. Our research will also synergize with graduate and undergraduate education for students from engineering, medical, and nursing schools, and will actively engage K-12 students and the general public. 

% \vspace{2 em}
% \noindent
% \textbf{Keywords} --- Customizability, Lowering Barriers, Learning, Human-Robot Interaction, Medical